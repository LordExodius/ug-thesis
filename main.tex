\documentclass[11pt]{article}
\usepackage[width=150mm]{geometry}
\usepackage{graphicx}
\usepackage{float}
\usepackage{amsmath}
\usepackage{listings}

% Import last
\usepackage{hyperref}

% Include graphics in answer
\newcommand{\nocapfigure}[2][0.5] {
    \begin{figure}[H]
    \centering
    \includegraphics[width=#1\linewidth]{#2}
    \end{figure}
}

\newcommand{\capfigure}[3][0.5] {
    \begin{figure}[H]
    \centering
    \includegraphics[width=#1\linewidth]{#2}
    \caption{#3}
    \end{figure}
}

% About
\title{The Effects of  Microservice-based Applications for Hybrid Cloud Edge Networks}
\author{Yan (Oscar) Yu}
\date{\today}

\begin{document}

\begin{titlepage}
    \begin{center}
        \vspace*{1cm}

        \Huge{Optimizing Microservice-based Applications for Cloud and Edge Architectures}
        \vspace{0.5cm}

        \Large
        Undergraduate Thesis
        \vspace{0.5cm}

        by
        \vspace{0.5cm}

        \textbf{Yan (Oscar) Yu}

        \vfill

        \large
        CS 4490Z\\
        Thesis Supervisor: Hanan Lutfiyya\\
        Course Instructor: Nazim Madhavi

        \vspace{0.5cm}
        Department of Computer Science\\
        Western University, London, Ontario N6A 5B7, Canada\\
        \today

    \end{center}
\end{titlepage}

\addcontentsline{toc}{section}{Abstract}
\section*{Abstract}

Edge computing has been the subject of much attention in the software development space over the 
last several years as the limitations of traditional cloud computing models continue to be exposed 
by an increasing number of connected IoT and internet-enabled devices that require real-time 
computing. As this new computing paradigm becomes more prevalent in the industry, it is important 
that software is developed effectively to take advantage of the benefits that edge computing 
brings to the table.
\newline

In this study, we attempt to establish an understanding of performance patterns that will enable the 
effective design and development of distributed software systems that can be easily deployed and 
optimized for various architectures of computing models -- primarily hybrid cloud edge networks.
\newline

% \newpage
% \addcontentsline{toc}{section}{Acronyms}
% \section*{Acronyms}
% \begin{itemize}
%     \item \textbf{HCE} Hybrid Cloud Edge
%     \item \textbf{WS} Web Server
%     \item \textbf{IS} Inference Service
%     \item \textbf{SSD} Single Shot Detection
%     \item \textbf{CNN} Convolutional Neural Network
%     \item \textbf{E2E} End-to-end
%     \item 
% \end{itemize}

\newpage
\tableofcontents

\newpage
\section{Introduction}
\subsection{Edge Computing}
Edge computing is a computing paradigm in which compute and data operations live on and are cached
on the network edge, in addition to traditional data caching provided by CDNs in the cloud computing
paradigm \cite{kubiak_possible_2022}. Findings show that the main advantage of the edge computing model is 
to reduce the time required to transfer large amounts of data to and from the cloud, as the 
amount of data produced by IoT and other devices on the network edge continues to grow every year
\cite{7488250}.
This enormous amount of raw data being generated highlights weaknesses in the traditional cloud
computing paradigm, as the bandwidth and network infrastructure required to transfer this becomes 
increasingly great.\newline

For real time applications like video analytics, 
where the application demands the real-time ingest and processing of high-definition video sources \cite{NAYAK2024783},
proposals have long been made to bring the cloud closer to the source devices \cite{satyanarayanan_case_2009}; thus reducing the
necessary bandwidth and network latency of the system overall.\newline

However, tradeoffs of bringing these operations to the network edge include limited computing 
resources, storage, and processing power. In addition, modern software applications are often
hosted remotely on servers and compute units operated by large corporations such as Amazon's AWS, 
Google's GCP, or Microsoft's Azure through a PaaS (Platform-as-a-service) model, with edge
resources being significantly more expensive than equivalent cloud offerings.

\subsection{Microservice Architecture}
Microservices are autonomous services deployed independently of each other, each performing some
specific operation towards a single defined purpose \cite{auer_monolithic_2021}. Services can be developed and deployed
independently of each other, maintaining only a shared communication interface. These properties
make microservices ideal for deployment to distributed networks consisting of various edge
or cloud nodes, as services can be independently deployed to the hardware or host that most 
optimally supports its performance characteristics without affecting other parts of the 
software system. These services are enabled by the use of containers or virtual machines (VMs)
to support hardware agnostic deployments. \newline

\newpage
\section{Research Objectives}
\begin{itemize}
    \item [(O1)] {
        Understand what applications may benefit from deployment to edge networks
        }
    \item [(O2)] {
        Understand how to identify performance characteristics of individual components within
        a software application
        }
    \item [(O3)] {
        Understand how applications can be developed to optimize for deployment across different 
        network architectures
        }
    \item [(O4)] {
        Understand how various network architectures can affect latency and user experience
        }
\end{itemize}

\newpage
\section{Methodology}
\subsection{Application Development}
Not all software is created equally, with a vast range of performance characteristics and
behaviours across various applications when developed for differing use cases. For the purposes
of identifying favourable conditions for edge deployments versus traditional cloud architectures, 
we developed a minimal microservice-architecture application for real-time object recognition. 
This design is an ideal representation of an edge-native application, demonstrating simultaneous 
requirements for high network bandwidth, low-latency interaction, and intensive compute across 
the application's various components. (O1)\newline

\capfigure[0.75]{images/applicationdesign1}{Sample application architecture for object recognition.}

As shown Figure 1, the application is composed of a web interface, API gateway/web server, 
and the inference service. The web server and inference service were containerized using Docker 
in order to enable configurable, platform agnostic deployment. (O3)\newline

The three components depend heavily on network bandwidth, as the client passes image data to the
web server, which forwards it to the inference service for processing. Despite preprocessing on the 
client side to downscale snapshot resolution to 640 x 480, each request to the web server contains 
over 20 kilobytes of image data. (O2) \newline

In a realistic scenario, 1080p real-time video analysis will require anywhere from 24-120 requests 
per second. With a 1920x1080 snapshot being around 2 megabytes, this results in a lower bound for 
bandwidth of of 384 Mbps, or an upper bound of 1.92 Gbps. As a result of this bandwidth bottleneck,
this type of application is an extremely good candidate for edge deployments \cite{7488250}, and could see 
significant improvements in various performance metrics versus a traditional cloud paradigm. (O1, O2)

\subsubsection{Inference Service}
To enable interfacing with a pre-trained object detection model, we leverage the open source
TensorFlow Serving infrastructure to expose an interface for performing inference with the model 
through a REST API or gRPC calls. This is a production ready system that introduces minimal
additional latency, making it suitable for the sample application.\newline

The object detection model chosen for the sample application is a pre-trained copy of SSD 
MobileNet v2, a multi-layer object detection model based on Google's MobileNetV2 CNN \cite{sandler_mobilenetv2_2019} and the SSD 
(Single Shot MultiBox Detector) \cite{liu_ssd_2016}. Sourced from the TensorFlow Model Zoo, this was trained on the 
COCO 2017 dataset. This model was chosen for its small size and ability to run efficiently on lower
spec CPUs, as compute power and memory are often constrained on edge devices or servers.
\newline

Despite the limited scale of this model, inference through a CNN (Convolutional Neural Network) 
remains a compute intensive task that will likely be a performance bottleneck that scales with 
hardware performance on the host device, as the process relies on large numbers (see: billions) 
of parallel matrix operations \cite{copeland_whats_2016}. (O2)\newline

A note regarding the configuration of this service: while the option to run the service with GPU 
acceleration was available, the CPU-only deployment was chosen to reduce additional variables 
as a result of differing hardware configurations locally and on the cloud host.\newline

Interfacing with the inference service was done through REST API calls, passing in image data
through an HTTP POST request to the \lstinline|/predict| endpoint with a body containing a 
preprocessed pixel array containing RGB colour channel values. 

\subsubsection{Web Server/API Gateway}
The web server and API gateway functionality for this application are combined into a single
package within this component. Built using Python on the FastAPI web framework, the web server
accepts incoming HTTP POST requests to the \lstinline|/image| endpoint. The logic for this endpoint 
is as follows:

\begin{enumerate}
    \item {
        Preprocesses the image using the Python PIL and NumPy libraries to parse the image data 
        into a correctly shaped tensor represented by a multi-dimensional array, which is then 
        sent to the inference server for object detection.
    }
    \item {
        After receiving a response from the inference server, the web server parses the JSON 
        response and filters out objects with a confidence score of less than a set threshold; 
        0.5 for the tests performed for data collection.
    }
    \item {
        The remaining objects are packaged back into JSON format before being sent back in the
        response to the POST request.
    }
\end{enumerate}

\subsubsection{Client Application}
The client application is a basic JavaScript application that uses WebRTC to capture the input
video source from a connected device. It accepts video inputs of size 640x480 to 1920x1080, and 
uses the Canvas API to take snapshots of the input and downsize them to 640x480 before sending
them to the web server via HTTP POST request. It parses the object data returned by the web server
to draw bounding boxes over the video input as a graphical representation of the object detection
process.\newline

\noindent The source code for these components are located on Github in the following repositories:
\begin{enumerate}
    \item \url{https://github.com/LordExodius/HCE-inference-server}
    \item \url{https://github.com/LordExodius/HCE-web-server}
    \item \url{https://github.com/LordExodius/HCE-client}
\end{enumerate}

\subsection{Deployment Architecture(s)}
To demonstrate the feasibility of developing an application compatible with multiple network 
architectures and gather data regarding the performance of an identical application on different
architectures, the web server and inference service were deployed in three configurations across 
a cloud host and locally, to emulate an edge server. (O3)

\begin{enumerate}
    \item {
        Local Web Server \& Local Inference Service. This configuration represents a full edge 
        deployment, with minimal geographical distance and high bandwidth between all components.
        The web server and inference service shared a resource pool consisting of an AMD Ryzen 9 5900X
        with 12 cores and 16GB of memory allocated to Docker.
    }
    \item {
        Local Web Server \& Remote Inference Service. This configuration represents a hybrid
        cloud edge deployment, with the inference service being deployed via Docker image to a
        Heroku instance running on a single Performance-L Dyno, with 8 CPU cores and 14GB of memory.
    }
    \item {
        Remote Web Server \& Remote Inference Service. This configuration represents a cloud native
        deployment, with everything except the client running on independent Heroku instances.
        Both component are running on Performance-L Dynos, with access to 8 CPU cores and 14GB of
        memory each.
    }
\end{enumerate}

\newpage
\section{Results}
\subsection{Data}
The graphs below show the latency data collected from deploying the sample application across 3
different configurations of cloud and local hosting, which was used as a proxy for an edge server 
given the time and resource constraints during the data collection stage. \newline

End-to-end (E2E) latency as shown here can also be interpreted as frametimes in a graphics processing 
context, as it also represents the time since the last frame drawn for the detection box overlay.
We can invert it to calculate frames per second (FPS), which may be a more intuitive metric for
how "smooth" or performant the system is:
$$
\text{FPS}=\frac{1000\text{ms}}{\text{E2E Latency (ms)}}
$$

\capfigure[]{images/latency1}{Graph showing E2E latency of a single 
video frame starting from the application client over 1000 samples. 
(WS: Web Server; IS: Inference Service)}
Latency values were collected from the client application by calling the \lstinline|performance.now()|
method to return a high resolution timestamp in milliseconds prior to each web server request,
and immediately after drawing the detection boxes on the output canvas. As such, this measurement 
includes time taken to capture a frame snapshot, the inference request, and the detection box rendering.

\subsection{Latency Differences}
\capfigure[]{images/latency3}{Common benchmarking performance metrics for real time graphics.}
In Figure 3, we show several metrics for the latency samples collected from the three
configurations of deployments described previously. These correspond to common graphics benchmarking
metrics used for video games - another application in which users are highly sensitive to latency \cite{liu_effects_2023}.
\newline

Percentile values were generated in Microsoft Excel using the \lstinline|PERCENTILE.EXC()| 
function to exclude the first and last values from the 1000 samples collected, as the first frame
sample appeared to be non-representative of the set as a result of having to establish the 
initial WebRTC connection from camera to client and the Session object for TCP connection between 
the web server and inference service.\newline

99th and 99.9th percentile values help illustrate consistency in E2E latency or frame timing,
as they represent a reasonable lower bound for expected latency in the system. Large 
deviations between average and 99 percentile values indicate the presence of stutters or spikes
in latency that result in much longer frame times, degrading user experience. We can observe these
visually in Figure 1 as large vertical spikes, particularly in the Local WS + Remote IS deployment.

\newpage
\section{Discussion}
\subsection{Implications}
These results can be used as a starting point to optimize real-time, microservice based applications.
The specific latency results and performance data acquired from the testing performed are primarily
applicable to real-time video analysis, particularly for those using machine learning inference 
in order to perform image or object recognition. \newline

From these observations of the latency data, there appears a clear connection between the latency 
and the network architecture of an application -- especially in the case where the performance
characteristics are skewed towards network I/O operations and bandwidth. We can use similar
statistical metrics to further analyze other software applications, identifying the optimal or
most performant network architectures for any specific use-case or business requirement.

\subsection{Limitations and Generalizations}
Lack of flexibility in compute resources (see: no cloud GPU/TPU acceleration) limited the ability of
this study to model common scenarios where traditional cloud computing offers similar or greater
compute power at a reduced cost. With additional options for hardware acceleration, we could 
perform similar tests with varying levels of computing power for each component to provide a 
significantly more nuanced analysis of how to extrapolate an optimal network architecture based on 
various server configurations.\newline

Additionally, the test results in this study may have been impacted by the use of a local deployment
in lieu of a dedicated edge compute server. By deploying the local components to an actual edge
service like AWS Wavelength, we could rerun the tests to greatly improve the accuracy and applicability
of the results.\newline

The results of this study can be generalized to other software applications utilizing CNNs to perform
video analysis, and the conclusions reached regarding the effects of network architecture
on latency can be used to support the optimization of other distributed microservice-architecture 
applications. Testing performed in this study can be repeated for any containerized microservice 
to determine the effect of various network architectures on the E2E latency of the system to 
support the conclusions found here.

\newpage
\section{Conclusions and Future Work}
In performing this study, we aimed to identify factors that contributed towards the performance
of a software application when deployed to the cloud and edge devices in various configurations.
Objectives included identifying the types of applications that benefit from edge deployments and
the performance characteristics of those applications (O1, O2); understanding the development
process of an application designed for deployment across different network architectures (O3); and
understanding how combinations of cloud and edge deployments can affect latency and user experience (O4).
\newline

Key results and conclusions:

\begin{enumerate}
    \item {
        Not all applications are equal. Performance characteristics such as heavy bandwidth
        and CPU resource dependencies can greatly impact the influence of network architectures
        on overall latency.
    }
    \item {
        E2E latency of a networked system is heavily influenced by the proximity and latency of 
        bandwidth heavy components. This is supported by the latency values being measurably lower 
        for configurations where the web server component and inference service were deployed on 
        a shared network, despite a lower bandwidth connection to the client in the full cloud 
        deployment scenario.
    }
    \item {
        Edge deployments exhibit significantly lower and more consistent latency results than 
        cloud solutions, supporting the conclusion that edge computing offers significant advantages
        over a cloud computing paradigm, especially for real-time applications such as video
        or image analytics.
    }
    \item {
        Network architecture optimization can be an effective tool to improve the performance of
        a software system, as supported by the significant differences in latency seen between
        the edge native, hybrid cloud edge, and cloud native deployments.
    }
\end{enumerate}

Future extensions to the testing and research conducted in this thesis include:
\begin{itemize}
    \item {
        Incorporating additional performance metrics such as \emph{motion-to-photon} latency \cite{9229154}.
        Additional metrics such as these help provide a more accurate picture of how the user
        experience is affected by changes in network architecture.
        } 
    \item {
        Perform testing on commercial edge computing platforms such as AWS Wavelength to identify
        whether local deployments can be an effective proxy for the edge during application 
        development and optimization.
    }
\end{itemize}

\newpage
\section{Reference List}
\bibliographystyle{plain}
\bibliography{sources}

\end{document}