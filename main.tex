\documentclass[11pt]{article}
\usepackage[width=150mm]{geometry}
\usepackage{graphicx}
\usepackage{float} 

% Include graphics in answer
\newcommand{\nocapfigure}[2][0.5] {
    \begin{figure}[H]
    \centering
    \includegraphics[width=#1\linewidth]{#2}
    \end{figure}
}

\newcommand{\capfigure}[3][0.5] {
    \begin{figure}[H]
    \centering
    \includegraphics[width=#1\linewidth]{#2}
    \caption{#3}
    \end{figure}
}

% About
\title{Optimizing Microservice-based Applications for Cloud and Edge Networks}
\author{Yan (Oscar) Yu}
\date{\today}

\begin{document}

\begin{titlepage}
    \begin{center}
        \vspace*{1cm}

        \Huge{Optimizing Microservice-based Applications for Cloud and Edge Networks}
        \vspace{0.5cm}

        \Large
        Undergraduate Thesis
        \vspace{0.5cm}

        by
        \vspace{0.5cm}

        \textbf{Yan (Oscar) Yu}

        \vfill

        \large
        CS 4490Z\\
        Thesis Supervisor: Hanan Lutfiyya\\
        Course Instructor: Nazim Madhavi

        \vspace{0.5cm}
        Department of Computer Science\\
        Western University, London, Ontario N6A 5B7, Canada\\
        \today

    \end{center}
\end{titlepage}

\addcontentsline{toc}{section}{Abstract}
\section*{Abstract}

Edge computing has been the subject of much attention in the software development space over the 
last several years as the limitations of traditional cloud computing models continue to be exposed 
by an increasing number of connected IoT and internet-enabled devices that require real-time 
computing. As this new computing paradigm becomes more prevalent in the industry, it is important 
that software is developed effectively to take advantage of the benefits that edge computing 
brings to the table.
\newline

In this paper, we attempt to establish an understanding of core principles that will enable the 
effective design and development of distributed software systems that can be easily deployed and 
optimized for various architectures of computing models -- primarily hybrid cloud edge networks.
\newline

\newpage
\tableofcontents

\newpage
\section{Introduction}

\newpage
\section{Background and Related Work}
This thesis is based on existing work in multiple fields, including 
\subsection{Cloud Computing}
\subsection{Edge Computing}
\subsection{Microservice Architecture}

\newpage
\section{Research Objectives}
\begin{itemize}
    \item [(O1)] {
        Understand what applications may benefit from deployment to edge networks
        }
    \item [(O2)] {
        Understand how to identify performance characteristics of individual components within
        a software application
        }
    \item [(O3)] {
        Understand how applications can be developed to optimize for deployment across different 
        network architectures
        }
    \item [(O4)] {
        Understand how various hosting patterns can affect latency and user experience
        }
\end{itemize}

\newpage
\section{Methodology}
\subsection{Application Development}
Not all software is created equally, with a vast range of performance characteristics and
behaviour across various applications when developed for differing use cases. For the purposes
of demonstrating this behaviour in order to identify favourable conditions for edge deployments, 
we develop a minimal microservice-architecture application for real-time object recognition 
consisting of a web interface, API gateway/web server, and an inference service. (O1)\newline

\capfigure[0.75]{images/applicationdesign1}{Example architecture}

Real-time object recognition using a CNN like

\subsection{Deployment Architecture(s)}
To demonstrate the feasability of developing an application compatible with multiple network 
architectures and gather data regarding the performance of an identical application across multiple
architectures, we

\subsection{Performance Analytics}

\newpage
\section{Results}
\subsection{Test Application}
\subsection{Data}
The graphs below show the latency data collected from deploying the sample application across 3
different configurations of cloud and simulated edge compute servers
\nocapfigure[]{images/latency1}
\nocapfigure[]{images/latency2}
\subsection{Latency Differences}

\newpage
\section{Discussion}
\subsection{Implications}
\subsection{Limitations and Generalizations}
- Hardware limitations
- Limited model selection
- Relatively simple system design
- Lack of edge deployments
- No service level latency monitoring
- Lack of cloud hardware control 
- Only latency measurements, no hardware usage or identification of other bottlenecks

\newpage
\section{Conclusions and Future Work}
- Perform more analysis with more performance analytics

\newpage
\section{Reference List}

\end{document}