\documentclass[11pt]{article}
\usepackage[width=150mm]{geometry}
\usepackage{graphicx}

% Include graphics in answer
\newcommand{\nocapfigure}[2][0.5] {
    \begin{figure}[H]
    \centering
    \includegraphics[width=#1\linewidth]{#2}
    \end{figure}
}

\newcommand{\capfigure}[3][0.5] {
    \begin{figure}[H]
    \centering
    \includegraphics[width=#1\linewidth]{#2}
    \caption{#3}
    \end{figure}
}

% About
\title{Designing Microservice-based Applications for Hybrid Cloud Edge Networks}
\author{Yan (Oscar) Yu}
\date{\today}

\begin{document}

\begin{titlepage}
    \begin{center}
        \vspace*{1cm}

        \Huge{Optimizing Microservice-based Applications for Cloud and Edge Networks}
        \vspace{0.5cm}

        \Large
        Undergraduate Thesis
        \vspace{0.5cm}

        by
        \vspace{0.5cm}

        \textbf{Yan (Oscar) Yu}

        \vfill

        \large
        CS 4490Z\\
        Thesis Supervisor: Hanan Lutfiyya\\
        Course Instructor: Nazim Madhavi

        \vspace{0.5cm}
        Department of Computer Science\\
        Western University, London, Ontario N6A 5B7, Canada\\
        \today

    \end{center}
\end{titlepage}

\addcontentsline{toc}{section}{Abstract}
\section*{Abstract}

Edge computing has been the subject of much attention in the software
development space over the last several years as the limitations of traditional cloud computing 
models continue to be exposed by an increasing number of connected IoT and internet-enabled 
devices that require real-time computing. As this new computing paradigm becomes more prevalent 
in the industry, it is important that software is developed effectively to take advantage of the
benefits that edge computing brings to the table.
\newline

In this paper, we attempt to establish an understanding of core principles that will enable the 
effective design and development of distributed software systems that can be easily deployed and 
optimized for various configurations of computing models -- primarily hybrid cloud edge networks.
\newline

\newpage
\tableofcontents

\newpage
\section{Introduction}

\newpage
\section{Background and Related Work}

\newpage
\section{Research Objectives}

\newpage
\section{Methodology}

\newpage
\section{Results}

\newpage
\section{Discussion}

\newpage
\section{Conclusions}

\newpage
\section{Reference List}

\end{document}